\documentclass{article}

%% This is written as a latex document
%% Scott Saunders, 2017, 05, 02


\usepackage{hyperref}


\begin{document}

\section{Goal}

Find systems for altering UV mapping, their algorithms, etc. In mesh lab, (and possibly other) projects.

\section{Rational}

To help us gain an understanding for possible systems of mapping textures to 3D models, as well as gaining new insight on how to tackle improving texture quality.

\section{Findings}

\subsection{UV-mapping file format}
For a .obj (wavefront .obj), \href{https://en.wikipedia.org/wiki/Wavefront_.obj_file}{(Wikipedia)}, the texture is represented as a set of vertices, a set of texture coordinates (u,v) between 0 and 1, and a list of face elements.

\subsection{iseez3d}
Iseeze3d simply uses the vertices, vertex-textures points, and faces to generate its UV-map textures. The weird blurry effects found in Iseeze3D aren't actually used. My guess is that it is left in to permit some level of edge-blurring in the image.

\subsection{enhancement}
In theory, each face could be mapped into a triangular texture, without any loss of quality. The reason there are people manipulating the mapping is for better user editability.

%The only case for enhacnment is the idea of keeping the face togeather, however I'm not sure this will actually assist in any mannor... The only case it could would be in the interperlation between verticies, where if there was a span of black it would show such span in the model. However I believe this is usually handeld by the UV-islands.
 

%There are multiple ways UV-map an image, (this can been seen in blender).
%
%From a \href{https://sourceforge.net/p/meshlab/discussion/499533/thread/93406a9f/?limit=25}{post}, there is a reference to using meshlab (and blender) to do UV mapping...
%
%% A UV layout per texture
%% A raw model
%% Iso parameterization (Likly best, from post above0
%% Remeshing, SImplification and Reconstrctioin | Iso Parameterization (Abstract Domain)
%% above, but Atlased mesh (Uses abstract domain to generate actual UV layout. Will allso modify the mest.)
%% In short: See texture menu.
%
%% Texture backing/merging: 
% Import model. To transfer the textures from the original to your new UV mapping (target), you want to use Texture, Transfer vertex attributes to texture (between 2 meshes). For Color data source select "Texture color." the rest of the optiuons are nice.

%12:30pm - Trying to figure out what the blurry stuff is in itseez3d... 
%1:13pm - Iseez3d just simply has this as extra data that they don't remove. In meshlab there is an ability to view the UV-texture mapping plane.


\end{document}
