%% This is written as a latex document
%% Scott Saunders, 2017, 05, 02


\usepackage{hyperref}


\begin{document}

\section{Goal}

Find systems for altering UV mapping, their algorithsm, ect. In meshlab, (and possibly other) projects.

\section{Rational}

To help us gain an understanding for possible systems of mapping textures to 3D models, aswell as gaining new insight on how to tackle improving texture quality.

\section{Findings}

For a .obj (wavefront .obj), \href{https://en.wikipedia.org/wiki/Wavefront_.obj_file}{(wikipedia)}, the texture is represented as a set of verticies, a set of texture coordinates (u,v) between 0 and 1, and a list of face elements. 

%There are multiple ways UV-map an image, (this can been seen in blender).
%
%From a \href{https://sourceforge.net/p/meshlab/discussion/499533/thread/93406a9f/?limit=25}{post}, there is a reference to using meshlab (and blender) to do UV mapping...
%
%% A UV layout per texture
%% A raw model
%% Iso parameterization (Likly best, from post above0
%% Remeshing, SImplification and Reconstrctioin | Iso Parameterization (Abstract Domain)
%% above, but Atlased mesh (Uses abstract domain to generate actual UV layout. Will allso modify the mest.)
%% In short: See texture menu.
%
%% Texture backing/merging: 
% Import model. To transfer the textures from the original to your new UV mapping (target), you want to use Texture, Transfer vertex attributes to texture (between 2 meshes). For Color data source select "Texture color." the rest of the optiuons are nice.

%12:30pm - Trying to figure out what the blurry stuff is in itseez3d... 
%1:13pm - Iseez3d just simply has this as extra data that they don't remove. In meshlab there is an ability to view the UV-texture mapping plane.


\end{document}
